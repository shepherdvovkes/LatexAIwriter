\documentclass{article}
\usepackage{tikz}
\usepackage[utf8]{inputenc}
\usepackage{geometry}
\geometry{margin=1in}

\title{Потоки данных в рое дронов}
\author{ }
\date{}

\begin{document}

\maketitle

\section*{Краткий обзор системы}

Разрабатываемая система представляет собой рой из 100 автономных дронов, организованных в трёхуровневую иерархическую структуру. Основная задача — проведение разведки, картографирование и передача данных в условиях отсутствия радиочастотной связи, с минимальной уязвимостью к обнаружению. Для обеспечения эффективного взаимодействия между уровнями используются лазерные и инфракрасные технологии связи с приоритетом прямой видимости. 

Каждый уровень выполняет специализированные функции:
\begin{itemize}
    \item \textbf{Уровень 1 (Командный)} — высокомощные дроны с LIDAR-сенсорами и средствами связи с базой.
    \item \textbf{Уровень 2 (Связной)} — промежуточные дроны-ретрансляторы и маяки навигации.
    \item \textbf{Уровень 3 (Рабочий)} — малые дроны, отвечающие за разведку на местности.
\end{itemize}

Система также способна интегрироваться с внешними FPV-ударными дронами, которым оперативно передаются разведданные для последующего принятия решений о поражении целей.

\section*{Интеграция с системами НАТО и Украины}

Для обеспечения совместимости с существующей архитектурой НАТО и ВСУ, система роя должна поддерживать следующие интерфейсы, протоколы и стандарты:

\subsection*{Тактические и стратегические системы НАТО}
\begin{itemize}
    \item \textbf{Link 16} — тактическая цифровая сеть передачи данных. Интеграция через шлюзы с поддержкой IP-туннелирования или JREAP.
    \item \textbf{JREAP-C} — расширение Link 16 через IP. Возможность ретрансляции разведданных с командных дронов (уровень 1) через наземную станцию.
    \item \textbf{STANAG 4586} — стандарт взаимодействия с БПЛА. Поддержка интерфейса обмена командами и телеметрией.
    \item \textbf{STANAG 4609 / 4545} — форматы видео- и разведданных (ISR). Используются для передачи потоков с камер и лидаров.
    \item \textbf{FMN (Federated Mission Networking)} — общая миссионная сеть НАТО. Поддержка федеративной маршрутизации и метаданных.
    \item \textbf{MAJIIC II / MAJIIC III} — архитектура многонационального обмена разведданными. Возможность автоматической маршрутизации целевых координат и событий.
\end{itemize}

\subsection*{Системы ВСУ и украинские боевые интерфейсы}
\begin{itemize}
    \item \textbf{Delta (Дельта)} — C4ISR-платформа украинского производства. Требуется REST API, JSON-совместимый, с передачей координат целей и карт.
    \item \textbf{Kropyva (Кропива)} — цифровая система управления артиллерией. Интеграция возможна через экспорт координат целей в формате UTM/LL.
    \item \textbf{GIS Arta / ArtOS} — геоинформационные системы огневого управления. Поддержка передачи разведданных по UDP/TCP через API шлюзы.
    \item \textbf{REDCON / BMS платформы} — тактические интерфейсы управления. Поддержка взаимодействия через MQTT/BROKER слои и защищённые API.
    \item \textbf{FPV-контроллеры нового поколения} — взаимодействие через мобильные командные станции или роевые хабы, передающие целеуказание в реальном времени.
\end{itemize}

\textbf{Примечание:} Все интерфейсы должны быть реализованы через трансляторы, шлюзы или контейнеризированные API-прокси, размещённые на командных дронах или наземных станциях. Приоритет — совместимость, масштабируемость и безопасность.

\section*{Заключение}

Документ завершён. Все блоки LaTeX корректно открыты и закрыты. Готов к компиляции и расширению. При необходимости можно добавить блоки с кодом API, архитектурными схемами и примерами взаимодействия с боевыми системами.


% --- BEGIN GATEWAY SECTION ---
\section*{Архитектура шлюзов и API-взаимодействие}
\subsection*{Схема шлюзовой архитектуры}
\begin{center}
\begin{tikzpicture}[node distance=2.5cm, every node/.style={draw, align=center, minimum width=3.5cm, minimum height=1.2cm}]
\node (drone) {\textbf{Командный дрон (Ур. 1)}};
\node[right=of drone] (gateway) {\textbf{Универсальный шлюз}};
\node[above right=of gateway] (delta) {\textbf{Delta API}};
\node[right=of gateway] (kropyva) {\textbf{Kropyva Interface}};
\node[below right=of gateway] (link16) {\textbf{Link 16 / JREAP}};
\node[below=of gateway] (gisarta) {\textbf{GIS Arta UDP API}};
\draw[->] (drone) -- node[above] {Данные, цели, изображения} (gateway);
\draw[->] (gateway) -- (delta);
\draw[->] (gateway) -- (kropyva);
\draw[->] (gateway) -- (link16);
\draw[->] (gateway) -- (gisarta);
\end{tikzpicture}
\end{center}

\subsection*{Примеры API-запросов}

\paragraph{Delta API — JSON REST пример:}
\begin{verbatim}
POST /api/v1/targets HTTP/1.1
Host: delta.local
Content-Type: application/json
Authorization: Bearer <token>

{
  "target_id": "drone_4231",
  "lat": 48.3794,
  "lon": 31.1656,
  "type": "vehicle",
  "confidence": 0.92,
  "source": "swarm_uav"
}
\end{verbatim}

\paragraph{Kropyva — передача координат в формате UTM:}
\begin{verbatim}
{
  "utm_zone": "36N",
  "easting": 345210,
  "northing": 5032123,
  "target_type": "infantry",
  "accuracy_m": 10
}
\end{verbatim}

\paragraph{GIS Arta — UDP пакет передачи цели:}
\begin{verbatim}
HEADER: ARTOS
PAYLOAD:
{
  "id": "obj_101",
  "coordinates": [48.40, 31.12],
  "classification": "tank",
  "priority": 1
}
\end{verbatim}

\paragraph{Link 16 / JREAP-C — конвертация целевых координат:}
\begin{verbatim}
[TargetReport]
ID = 9032
Latitude = 48.3794
Longitude = 31.1656
TrackQuality = 5
Classification = GroundVehicle
TimeStamp = UTC2025-06-15T12:05:32Z
\end{verbatim}

% --- END GATEWAY SECTION ---
\end{document}
